\chapter{Conclusion}\label{chapter_conclusion} \label{ch6}

The starting point for our exploration in this monograph was the Emergent Grammar Hypothesis,\is{Emergent Grammar hypothesis} that there are no -- or minimal -- innate language-specific principles driving the shape of the adult morphophonological grammar.\is{innateness}


\begin{example} \et{The Emergent Grammar Hypothesis}\is{Emergent Grammar hypothesis} (Repeated  from box {\ref{box:box-EG-hypothesis}} on p.\ \pageref{box:box-EG-hypothesis})\\
General human cognition\is{cognition} provides much, if not all, of the necessary scaffolding for the acquisition\is{acquisition} of morphophonology, allowing construction of a phonological grammar\is{grammar} of the ambient language.
\end{example}

As outlined in Chapter \ref{intro_motivation}, Emergence assumes that the critical elements for acquiring a phonology are aspects of cognition that humans use in their general interactions with the world, not simply with respect to language -- attention\is{attention}, memory\is{memory}, similarity\is{similarity}, sequential processing\is{sequence!processing}\is{processing}, frequency\is{frequency}, categories\is{category}, generalising (including generalising over generalisations)\is{generalisation}, and identity.\is{identity}\is{Identity Principle} Chapter \ref{ch2} provided an extensive example from Yangben\il{Yangben} of how these general principles of cognition might result in the language learner acquiring\is{acquisition!constituent}\is{acquisition!morph} categories like ``segment'', ``feature'', \is{morph}``morph'', and ``word''.\is{word} In examining phonological patterns from this perspective, as is done in Chapter \ref{chapter_grammar}, we  proposed two general mechanisms for representing phonological patterns -- \is{morph set}morph sets and well-formedness conditions.\is{well-formedness condition} We summarise the role of each, repeating figures from Chapter \ref{chapter_grammar} for convenience.

Systematicity within minimal morph sets\is{morph set!minimal} is expressed with {\it Morph Set Relations} (MSRs) and {\it Morph Set Conditions} (MSCs). MSRs encode systematic relations between morphs in minimal morph sets, while MSCs, which govern when a minimal morph set is well-formed, result in the productive\is{productivity} expansion of morph sets.\is{morph set!minimal}


\begin{whiteshadowbox}\is{Morph Set Relation!formalised}
\begin{example}\et{Morph Set Relation (Chapter \ref{ch3}, (\ref{MSR-schema}))}\label{MSR-schema-final}\is{Morph Set Relation!formalised}

In a minimal morph set, there is a systematic relation between morphs with α\  (subject to $\mathcal{C}$\down{\it m}) and morphs with β\  (subject to $\mathcal{C}$\down{\it n}).\\

{MSR}: \begin{tabular}[t]{ll} \{$\mathcal{M}$\down{\it i}, $\mathcal{M}$\down{\it j}\} & $\mathcal{M}$\down{\it i}: α\ ($\land$ $\mathcal{C}$\down{\it m})\\
                               &$\mathcal{M}$\down{\it j}: β\ ($\land$ $\mathcal{C}$\down{\it n})\\
       \end{tabular}
\end{example} 
\end{whiteshadowbox}


\begin{whiteshadowbox}\is{Morph Set Condition!formalised}
\begin{example} \et{Morph Set Conditions (Chapter \ref{ch3}, (\ref{MSC-schema}))} \ee \is{Morph Set Condition!formalised}

\ea With respect to MSR\down{γ}, a minimal morph set is ill-formed if there is an $\mathcal{M}$\down{\it i} and there is no corresponding $\mathcal{M}$\down{\it j}.
    \begin{center}MSC\down{γ}: For $\mathcal{M}$\down{\it i}, $\mathcal{M}$\down{\it j} of MSR\down{γ}, *\{$\mathcal{M}$\down{\it i}, $\neg$$\mathcal{M}$\down{\it j}\}\end{center}~\\
    
\ex With respect to MSR\down{γ}, a minimal morph set is ill-formed if there is an $\mathcal{M}$\down{\it j} and there is no corresponding $\mathcal{M}$\down{\it i}.
    \begin{center}MSC\down{γ}: For $\mathcal{M}$\down{\it i}, $\mathcal{M}$\down{\it j} of MSR\down{γ}, *\{$\lnot$$\mathcal{M}$\down{\it i}, $\mathcal{M}$\down{\it j}\}\end{center}
\z
\end{example}
\end{whiteshadowbox}
\is{morph set!minimal}

Well-formedness is characterised by three classes of conditions, {\it type} conditions, {\it syntagmatic} conditions, and {\it paradigmatic} conditions. Where observed properties  occur with varying frequencies, type conditions penalise the infrequent properties. \is{well-formedness condition!type, formalised}

\begin{example} \et{Type condition schema (Chapter \ref{ch3}, (\ref{types-schema-original}))} \label{types-schema-revised}\is{well-formedness condition!type}\smallskip\\
\begin{tabularx}{\linewidth}{@{}lQ@{}}
*[X] 	& Assign a violation to a form for each [X], where [X] may be either morphological or phonological.
\end{tabularx}
\end{example}

Syntagmatic conditions, conditions governing sequences\is{sequence!phonotactics} of properties, characterise unattested/underattested sequential patterns of occurrence.\is{well-formedness condition!syntagmatic, formalised}

\begin{example} \et{Syntagmatic schema (Chapter \ref{ch3}, (\ref{phonotactics-schema-original}))}\label{phonotactics-schema-revised}\is{well-formedness condition!syntagmatic}\smallskip\\
\begin{tabularx}{\linewidth}{@{}lQ@{}}
*[X][Y]& Assign a violation to a form for each  sequence of [X] followed by [Y], where [X], [Y] may be either morphological or phonological.
\end{tabularx}
\end{example}


\hspace*{-1mm}Finally, asymmetries in the cooccurrence\is{cooccurrence} of observed properties motivate paradigmatic conditions.\is{well-formedness condition!paradigmatic}

\begin{example} \et{Paradigmatic schema (Chapter \ref{ch3}, (\ref{paradigmatic-schema-original}))} \label{phonotactics-schema-paradigmatic-final}\label{paradigmatic-schema-revised}\is{well-formedness condition!paradigmatic}\smallskip\\
\begin{tabularx}{\linewidth}{@{}lQ@{}}
{*}$\begin{bmatrix}\textrm{X}\\\textrm{Y}\end{bmatrix}$ & Assign a violation to a form for each combination of [X] and [Y], where [X], [Y] may be either morphological or phonological.
\end{tabularx}
\end{example}

These mechanisms are a formal means of representing the logical result of acquiring\is{acquisition} a phonological grammar\is{grammar} under Emergence;\is{innateness} they are not additional language-specific innate cognitive mechanisms.\is{cognition}

A further logical result, examined in Chapter \ref{chapter_URs} and tested against complex patterns in five languages in Chapter \ref{chapter_consequences}, is that there is no motivation or evidence for unique \is{underlying representation!unique}underlying forms/inputs, despite what is assumed in most phonological theories. We found no evidence for the kind of abstract ``underlying'' representations\is{abstractness!underlying representation}\is{underlying representation!abstractness} that are commonly assumed, finding instead that such underlying forms introduce analytic difficulties  and raise significant conceptual challenges.

It is from observed surface variety that structuralist\is{structuralism} frameworks, traditional generative phonology\is{generative phonology} and Optimality Theory\is{Optimality Theory} construct {\it underlying forms} and from which Emergent Phonology constructs {\it morph sets}. An argument for a unique \is{underlying representation!unique}underlying form might be that it provides an account of the systematicity in the relations observed between multiple surface forms. This argument, however, proves to be illusory. While accounting for the observed systematicity is important, we have shown that the systematicity can  be expressed through a network of \is{surface-to-surface}surface-based relations (Morph Set Relations and Morph Set Conditions) and phonological and morpho-phonological well-formedness conditions.\is{well-formedness condition}

In concluding this work, we highlight briefly a number of additional implications of Emergence for phonological systems, broadly construed. In some cases, the implications seem consistent with the predictions of Emergence while in others they may pose challenges.

\section{Phonological phenomena}

The examples discussed in this monograph merely skim the surface of the rich and diverse patterns of sounds in the world's languages. These patterns have led researchers to posit constructs such as directional spread, \is{iterativity}iterative and non-iterative application, radical underspecification,\is{underspecification} and a \is{Lexical Phonology}stratal organisation for the lexicon, targeting phenomena such as complex harmony patterns,\is{harmony} disharmony,\is{disharmony} \is{chain shift}chain shifts, and so on. Here we sketch briefly how Emergence might approach some of these phenomena.

\subsection{Directionality}

\is{directionality|(}Numerous phonological patterns exhibit apparent directional effects where it seems on the surface to make a difference whether the ``target'' of some process is to the left or right of the process's trigger. Many cases of this type have been discussed. In some cases, only targets on one side of a trigger are affected (for example, in Fula\il{Fula} (\citealt{Paradis:1992, Archangeli+:1994}; see \citealt{Archangeli+:OUP} for an Emergent account) mid vowels to the left of a high vowel are advanced while those to the right are unaffected), in some cases targets on both sides of a trigger are affected (e.g.\ bidirectional harmony\is{harmony!bidirectional}\is{harmony!Akan}\is{directionality!bidirectional} in Akan,\il{Akan} \citealt{Clements:1985geometry}), while in other cases potential targets are treated differently on one side or the other of a trigger (e.g., in Maasai,\il{Maasai}\is{harmony!Maasai} a low vowel to the left of a trigger blocks harmony while a low vowel to the right undergoes harmony, shifting to a mid vowel, \citealt{Levergood:1984}). 

To a certain extent, the directionality issue in an Emergent framework is comparable to much work in Optimality Theory.\is{Optimality Theory} If the source of harmony is an intrinsically nondirectional syntagmatic condition,\is{well-formedness condition!syntagmatic} as assumed here, then all directional effects should be derivative (\citealt{Bakovic:2000}). Unlike a contextually unconditioned \is{Agree constraint}``Agree'' constraint though, directional effects can be achieved by \is{directionality!syntagmatic condition} syntagmatic conditions or by regularities within morph sets. For example, a condition prohibiting mid retracted vowels before high vowels will cause mid vowels to be advanced before a high vowel in Fula\il{Fula}\is{harmony!Fula} ([ɓet-ir-dɛ] `to weigh'), but not after a high vowel (*[ɓet-ir-de]) (\citealt[87]{Paradis:1992}). In addition, it is possible that certain \is{directionality!morph set}morph sets and not others exhibit related pairs of morphs. We have seen \is{idiosyncrasy}idiosyncratic differences in our discussion of Mayak\il{Mayak} (\Sec\ref{section_Mayak_low_vowels}) but it is also possible to see systematic differences. For example, if all suffixes had harmonic pairs in their morph sets whereas prefixes were systematically singletons, then harmony would apply to suffixes but not to prefixes. Warlpiri\il{Warlpiri}\is{harmony!Warlpiri} (\citealt{Nash:1980, Simpson:1983}) is a particularly interesting test case since nouns and verbs appear to motivate quite different patterns of harmony. In \citet{Archangeli+:Warlpiri}, we show that these differences can be derived by having the same phonotactics\is{phonotactics} governing both nouns and verbs, but with different Morph Set Relations for the two \is{category}categories.\is{directionality|)}

\subsection{Noniterativity}\is{noniterativity|(}

We have argued in \textsection\ref{section_Kinande} that the standard ``noniterative spread''\is{noniterativity!Kinande Noniterative Association} analysis of Kinande\il{Kinande} High tones is better seen as a type of morphologically induced ``selection'':\is{morpho-phonotactics!selection}\is{selection!morpho-phonotactics} a lexically arbitrary class of morphs requires a preceding morph that ends on a High tone. Not all cases of apparent noniterativity involve such blatant morphological conditioning, however. In Lango \il{Lango}\is{harmony!Lango} (\citealt{Woock+:1979, Noonan:1992}), for example, apparent \is{harmony!noniterative}noniterative tongue root harmony\is{tongue root!Lango harmony} is triggered by a vowel bearing the harmonic value: [b\ipa{\`{ɔ}ŋó-wú}] `your dress' vs.\  [b\ipa{\`{ɔ}ŋ\'{ɔ}}] `dress' (\citealt[22]{Woock+:1979}), with the appropriate morph selected from the morph set \{b\ipa{\`{ɔ}ŋ\'{ɔ}}, b\ipa{\`{ɔ}ŋó}\}\down{\sc dress}. Consistent with \citet{Kaplan:2008}, we consider that such cases of apparent noniterativity require something other than  a simple harmony phonotactic.\is{phonotactics!harmony} Our working hypothesis is that apparent noniterative patterns result either from limited options in the relevant morph sets (for example, Lango might systematically have morph sets like \{b\ipa{\`{ɔ}ŋ\'{ɔ}}, b\ipa{\`{ɔ}ŋó}\}, not \{\ipa{b\`{ɔ}ŋ\'{ɔ}, bòŋó}\}), or there might be a morphological condition on the phonotactic such that harmony would only be enforced at a morph boundary.\is{noniterativity|)}

\subsection{Saltation and chain shifts}\is{saltation|(}\is{chain shift|(}

Two recurrent problems for ranked\is{ranking!OT} constraint-based systems are saltation (\citealt{Hayes+:2015}) and chain shifts (\citealt{Kirchner:1996}). The problem appears to be similar for both patterns. With saltation, we see cases where instances of A alternate with C, but where an intermediate category B does not alternate, (\ref{saltation_picture}). If C is ``better'' than A by some constraint hierarchy, then C should be better than B as well, so both /A/ and /B/ should be realised as [C]. For example, in \il{German}German, spirantisation of voiced velars  (\citealt{Ito+:2003German}) results in voiced [\ipa{ɡ}] alternating with voiceless [x] or [\c{c}] while voiceless [k] does not alternate.



\begin{example} \et{Saltation}\label{saltation_picture}\ee
\begin{tikzpicture}
\node at (0,0) (A) {A};
\node [right=2em of A] (B) {B};
\node [right=2em of B] (C) {C};
\draw[-{Triangle[]}] (A) to[bend left] (C);
\end{tikzpicture}
\end{example}


\noindent In chain shifts, we can distinguish between two types of cases. In the first, the situation is very similar to that seen for saltation, except that A shifts to B and B shifts to C. (The English\il{English} ``Great Vowel Shift'' (\citealt{Chomsky+:1968}) is an instance of a chain shift.)

\begin{example} \et{Chain shifts 1} \label{chain_picture_1}\ee
\begin{tikzpicture}
\node at (0,0) (A) {A};
\node [right=2em of A] (B) {B};
\node [right=2em of B] (C) {C};
\draw[-{Triangle[]}] (A) to[bend left] (B);
\draw[-{Triangle[]}] (B) to[bend left] (C);
\end{tikzpicture}
\end{example}

\noindent In an extreme case, the chain can bite its tail, so to speak, creating a circle. C could shift to A in (\ref{chain_picture_1}), or, in a \is{circle}circle involving only A and B, A shifts to B and B shifts to A. Such chains raise two problems: in (\ref{chain_picture_1}), if C is ``better'' than B, then why would A shift to B and not continue to C? In (\ref{chain_picture_2}), if B is ``better'' than A along some dimension -- motivating the shift from A to B -- then how do we motivate a shift from B to A? (A classic case  is  the Taiwanese\il{Taiwanese} tone circle;\is{tone circle} for an Emergent analysis see \citealt{Archangeli+:2016mm}.)

\begin{example} \et{Chain shifts 2} \label{chain_picture_2}\ee
\begin{tikzpicture}
\node at (0,0) (A) {A};
\node [right=2em of A] (B) {B};
\draw[-{Triangle[]}] (A) to[bend left] (B);
\draw[-{Triangle[]}] (B) to[bend left] (A);
\end{tikzpicture}
\end{example}

 In the Emergent framework that we have sketched here, these cases require analyses that are quite different from either rule-based analyses or optimising constraint-based analyses. For saltation (\ref{saltation_picture}), since A alternates\is{alternation!morph set} with C, there must be a morph set  \{A, C\}. Conversely, since B does not alternate, it is in a singleton morph set, \{B\}.  There is no inherent contradiction in having morph sets containing \{A, C\} -- productively\is{productivity} related by a Morph Set Condition\is{Morph Set Condition} -- and also having morph sets containing \{B\}. Phonological optimisation\is{optimisation} is the domain of a different component, namely the well-formedness conditions.\is{well-formedness condition!phonological optimisation} We may choose \{C\} in some context due to phonological optimisation, and yet, with another morph set, choose \{B\} in that same context simply because B has no counterpart that would be ``better'' along the scale in question. The situation is similar for  the sequential chain shift in (\ref{chain_picture_1}): morph sets contain either \{A, B\} or \{B, C\}; no morph set contains *\{A, B, C\}. Optimisation does the best that it can, choosing C over B in one case and B over A in the other, depending on the context and the conditions governing it.

On the other hand, the \is{circle}circular chain shifts involve morph sets containing both A and B, \{A, B\}. Again, the domain of operation of MSRs and MSCs is to create morph sets of a particular structure, in this case \{A, B\} morph sets. However, in this instance, where set \{A, B\}\down{α} has A, morph set \{A, B\}\down{β} has B, and vice versa. Selection cannot be purely phonological with circular chain shifts: the prediction is that these patterns must involve some morphotactic or syntactic element to characterise the pattern followed by each morph set (as is indeed the case in the classic Taiwanese\il{Taiwanese} case, for \is{chain shift|)}\is{saltation|)}example).


\subsection{Opacity} 

Finally, a different kind of challenge occurs in certain cases of opacity\is{opacity|(} (\citealt{McCarthy:1999, McCarthy:2007_hidden, Idsardi:2000}), already touched on in \Sec\ref{section_Polish}. To recall, opacity refers to an analysis where some critical element for the analysis is not observed at the surface. For example, Standard Yorùbá\il{Yor\`ub\'a|(} exhibits a robust pattern of tongue root harmony\is{harmony!Yor\`ub\'a|(}\is{tongue root!Yor\`ub\'a harmony} (\citealt{Bamgbose:1967, Awobuluyi:1967, Archangeli+:1989}). Mid vowels within native vocabulary consistently show the same tongue root position as a following nonhigh vowel: [\ipa{\=ekp\=o}] `oil', [\ipa{\={ɔ}b\`{ɛ}}] `soup', *[eCɛ], *[oCa], etc. (\citealt[177]{Archangeli+:1989}). Derived mid-mid sequences, however, do not invariably harmonise. For example, when we see \is{rhotic!Yor\`ub\'a}[r]-deletion and the resulting application of adjacent vowel \is{assimilation!Yor\`ub\'a}assimilation, the results can be disharmonic:\is{disharmony} [\ipa{\=erùkp\`{ɛ}}] $\sim$ [\ipa{\=eèkp\`{ɛ}}] `earth', *[\ipa{\=ɛ\`{ɛ}kp\`{ɛ}}] (\citealt[187]{Archangeli+:1989}). Just as in the \is{saltation}saltation and \is{chain shift}chain shift cases, the answer in an Emergent framework\is{Emergent Grammar} lies in the distinction between regularities within morph sets and the optimisation\is{optimisation} determined by phonotactic\is{phonotactics} well-formedness conditions\is{well-formedness condition!syntagmatic} (rather than in a set of ordered rules\is{rule-ordering} perhaps recapitulating the \is{diachrony}historical development of the pattern, the standard analysis of the Polish yers\il{Polish}\is{yers} discussed in \textsection\ref{section_Polish}). In the Standard Yorùbá\il{Yor\`ub\'a|)}  example, complexities in the formulation of the MSR\is{Morph Set Relation!Yor\`ub\'a} with and without the encoding of harmony lead to the simplest MSR being one that derives [\ipa{\=eèkp\`{ɛ}}], and not *[\ipa{\=ɛ\`{ɛ}kp\`{ɛ}}] (see \citealt{Archangeli+:2015_YVH} for discussion).\is{opacity|)} The result violates the harmony phonotactic, but the phonotactics do not directly drive patterns of alternation.\is{alternation}\is{harmony!Yor\`ub\'a|)}




\subsection{Summary}

There are numerous other sorts of cases to be considered where an Emergent framework either provides a different way of viewing phenomena or forces an analysis to move in a particular direction. Such cases  routinely involve \is{underspecification}underspecified lexical entries in harmony\is{harmony} systems, \is{locality}long-distance effects in consonant\is{harmony!consonant} harmony, and so on. Phonological interactions with other modules of grammar\is{grammar}, such as morphology, are by their nature quite different from standard approaches. See, for example, \citet{Kwak:2020} on a preliminary Emergent alternative to level-ordering in \il{Tsilhqot'in}Tsilhqot'in. 

A recurring theme in our brief discussion of extensions of Emergence is the separation of responsibility between MSRs,\is{Morph Set Relation} MSCs,\is{Morph Set Condition} and well-formedness conditions.\is{well-formedness condition} MSRs characterise relationships within morph sets and MSCs ensure productivity\is{productivity} of these relationships, while well-formedness conditions penalise ill-formed morphs and morph compilations.\is{compilation} Despite their formal simplicity, interactions among these components are consistent with the diverse patterns observed in natural language.



\section{Prosody}

\is{prosody|(}There is a range of prosodic phenomena that have not been discussed at all in this work, phenomena including the encoding of length, determining whether and how to include syllables,\is{syllable} how to represent \is{templatic phenomena}templatic phenomena, \is{reduplication}reduplication, and so on. The challenge in considering such phenomena is to determine both whether the sorts of structures that have been proposed are required in an Emergent framework and if so, how to derive them. It is important to consider two different types of properties. Some properties are ``concrete'', in the sense that they are directly encoded phonetically.\is{phonetics} For example, long vowels and consonants, tone, the actual manifestation of stress\is{stress} and intonation are present in the phonetic string encountered by a learner.\is{acquisition!tone}\is{tone!acquisition} In contrast, \is{constituency!prosodic}constituency -- \is{syllable}syllables, feet\is{foot}, prosodic words,\is{prosodic word} etc. -- are not directly encoded, but are postulated to explain a variety of patterns that are directly encoded:\is{acquisition!syllable}\is{acquisition!foot} we don't ``hear'' syllables or feet, but such constituents appear to determine whether a \is{reduplication}reduplicative form is well-formed or not, whether a consonant is released or not, flapped, voiced, and so on -- properties that are directly observable.\is{prosody|)}

\subsection{Syllables}

Consider syllable\is{syllable|(} structure, which we carefully set aside in box  {\ref{box:box-syllables}} on p.\ \pageref{box:box-syllables}. We know that segments may be realised very differently in ``onset'' and ``coda'' positions. Consider two tendencies identified in \citet[69]{Gick+:2006l}: ``(1) postvocalic liquids always have a measurable dorsal constriction; (2) patterns of gestural timing and magnitude in liquids are almost always different (asymmetrical) in pre- vs.\ postvocalic positions''. We could imagine a relatively abstract\is{abstractness} approach to such phenomena (\citealt{Kahn:1980, Clements+:1983, Levin:1985}; etc.) where phonological representations abstract away from the phonetic\is{phonetics} details, building representations that include syllable \is{constituency!prosodic}constituents that condition the phonetic realisation of such segments. The alternative is to build the phonetically different segments into our phonological representations, representing ``syllabically different'' segments in morph sets, establishing MSRs to relate them, and so on. This approach would relate more directly to work such as that of \citet{Steriade:1999syllabic}. We might assume, for example, that liaison in \il{French}French involves morph sets pairing a form without a consonant., e.g.\ \{\ipa{tʁo}\} `too much' with a form where the final consonant is marked as being obligatorily pre-vocalic, \{\ipa{tʁop\up{v}}\}. Syllabic structure involves a large number of interacting issues and patterns and we will not begin to address them here, noting only that the Emergent hypothesis\is{Emergent Grammar hypothesis} might lead one to explore those hypotheses that have depended on encoding more segmental properties than \is{constituency!prosodic}constituency-based properties.\is{syllable|)}

But does this mean that Emergence leads us away from constituency? What about \is{templatic phenomena}templatic morphology, \is{minimal word}minimal word considerations, \is{reduplication}reduplication, and so on? What about the role of \is{constituency!prosodic|(}constituency in patterns of rhythmic prominence?

\subsection{Stress, etc.}\is{stress}\is{prosody}

 There has been a great deal of attention in the literature on stress about whether constituency should factor into phonological analyses, for example, \citet{Prince:1983} for a grid-only approach, versus \citet{Halle+:1987} and \citet{Hayes:1995} for approaches incorporating constituency. Of particular importance, from the Emergentist perspective is work such as that incorporating the iambic-trochaic law (\citealt{Hayes:1985, Yiu:2018}). As Hayes points out, research in experimental psychology motivates trochaic groupings based on intensity and iambic groupings based on duration. Similar evidence can be found in music. So while it seems relatively uncontroversial that constituency plays a role in phonology, it seems much less clear that such constituency is specifically linguistic. This issue deserves much closer attention paid to it than we can do in these closing remarks. To be more comprehensive, an Emergent framework must address these issues of constituency in stress systems, in syllable structure\is{syllable} and then in a wide array of \is{templatic phenomena}templatic patterns, including\is{reduplication} reduplication. We will not engage in that discussion here but see this as an area ripe for future research. \is{constituency!prosodic|)}

\section{Beyond adult phonology}

Emergence has implications for a wide range of phonological phenomena, beyond characterising adult grammars.\is{grammar} For example,  the discussion of language acquisition in this monograph, particularly in Chapters \ref{ch2} and \ref{ch3}, is highly speculative, based on our understanding of the acquisition literature and on the logic of the Emergent hypothesis.\is{Emergent Grammar hypothesis} In this section, we touch on implications for a few such areas -- language change, \is{multilingualism}multilingualism, perception\is{perception} and frequency in production.\is{production}  

\subsection{Language change}\is{diachrony|(}

Because an Emergent grammar is built firmly on the basis of observed forms, Emergence is consistent with an Evolutionary understanding of language change (\citealt{Blevins:2004, Wedel:2006}).  Emergence adds to the \is{Evolutionary Phonology}Evolutionary framework by providing a formalism for analyses of earlier and later stages of a grammar,\is{grammar} allowing the researcher to pinpoint with some precision the nature of the change and the forces leading to a particular change. 

An Emergent phonology\is{Emergent Grammar} characterises alternations\is{alternation!Morph Set Relation} in terms of MSRs\is{Morph Set Relation} (productive or not) and well-formedness conditions.\is{well-formedness condition} In the event that one component is general and phonological but the other  is bound to specific morphs, we expect to see change with the morph-bound component, but not directly with the general component. A case in point is  \is{consonant mutation}initial consonant mutation in Irish\il{Irish} (\citealt{McCullough:2020}), where the phonologically regular mutation occurs in \is{idiosyncrasy}idiosyncratic morphosyntactic contexts;\is{morphosyntax} the modern-day usage involves a solid grasp of the phonological forms, but inconsistency in the form selected for a given morphosyntactic category.\is{category!morphosyntactic} \citet{McCullough:2020} provides an Emergent analysis of the pattern and discussion of the changes in progress.

There are other imaginable implications for language change. For instance, the Emergent framework predicts that \is{idiolect!mutually intelligible}different speakers may have divergent formal representations of the same patterns, yet there may be no functional distinction between the two. This raises the possibility of a language change which has an effect on a subpart of the population of speakers, due to the different formal representations possible for the language.\is{diachrony|)}


\subsection{Multilingualism}\is{multilingualism|(}

Most of the world's population is multilingual. How are multiple languages represented in the speaker's mind, how are they kept separate, and how do they interact? \citet{Alfaifi:2020phd} offers a very interesting study of diglossia\is{diglossia} in the Faifi\il{Faifi} region of southern Saudi Arabia where the local language, Faifi, is used at home and in informal settings while a version of Modern Standard Arabic\il{Modern Standard Arabic} is used in schools and in formal settings. \citet{Alfaifi:2020phd} argues that much of the phonology, lexicon,  and morphology is shared between the two, but there are both  sounds and morphs --  stems, affixes, and \is{templatic phenomena}templates -- that are specific to one language and cannot be used in the other. Alfaifi's Emergent analysis characterises these patterns in terms of morph sets with single members, such as \{dawla\}\down{\sc country}, unrelated members, such as \{\ipa{daʕdaʕa\down{\sc low}, kalaːm\down{\sc high}}\}\down{\sc talk}, as well as morph sets with systematically related members, such as \{\ipa{t\up{h}alb, kalb}\}\down{\sc dog}. Alfaifi proposes Morph Set Relations\is{Morph Set Relation!Faifi} which assign a context-marker \textsc{high} or \textsc{low} (following terms familiar in the diglossia\is{diglossia} literature) to morphs containing specific sounds ([t\up{h}] is a \textsc{low} sound, not a \textsc{high} sound), as well as MSRs which relate morphs: for instance, there is an MSR relating \textsc{low} [t\up{h}] and \textsc{high} [k], hence, \{\ipa{t\up{h}alb}\down{\sc low}, \ipa{kalb}\down{\sc high}\}\down{\sc dog}. The large number of morph sets whose members are unrelated show that such relations are not productive, arguing against MSCs connected with the MSRs. Alfaifi also draws on well-formedness conditions\is{well-formedness condition} which penalise disagreement among context-markers within a word.\is{word!domain} This prohibits \textsc{low} morphology combined with \textsc{high} stems and the converse, \textsc{high} morphology with \textsc{low} stems, where \textsc{low} and \textsc{high} are read off the labels and the labels themselves are in some cases arbitrary and in other cases systematic. In addition to sorting out the complexities of Faifi\il{Faifi} diglossia,\is{diglossia} \citet{Alfaifi:2020phd} provides a roadmap for exploring \is{multilingualism}multilingualism under Emergence.\is{multilingualism|)}

\subsection{No ``reverse engineering''}\is{reverse engineering|(}\largerpage

There is an important difference between a theory that records surface representations\is{surface-to-surface} in morph sets\is{reverse engineering!morph set} and a theory which posits abstract underlying representations.\is{abstractness!underlying representation}\is{underlying representation!abstractness} With morph sets, what is observed on the surface is directly represented in the structure of a morph set: each phonologically distinct surface form is directly represented in the morph set. In contrast, a theory of underlying representations involves reverse engineering. Based on the observed surface forms, and based on the rules and/or constraints, a unique \is{underlying representation!unique}underlying representation is postulated that will give the correct surface results. Essentially, the effect of the rules/constraints is undone to see what sort of form can be postulated from which the observed surface forms can be derived. Crucially, once posited, the derivation of surface forms from this abstract \is{reverse engineering!underlying representation}underlying representation is proposed to be blind to the results. That is, when applying rules or constraints to some underlying representation, the computation has no look-ahead knowledge of what the results should be. What is odd about this failure to use ``look-ahead'' knowledge is that the (hypothetically inaccessible) ``look-ahead'' forms are precisely those that the learner has actually encountered and has used to create the underlying representation.

The absence of look-ahead knowledge creates problems when phonologically identical \is{underlying representation}underlying representations behave differently with respect to rules\slash constraints:  knowledge of the surface form is needed to determine the appropriate computation for a given underlying form, yet that requires access to the result of the computation, not blindness. As seen in the examples in Chapter \ref{chapter_consequences}, lexical items may differ in \is{idiosyncrasy}idiosyncratic ways that are easily identifiable on the surface, but that do not straightforwardly allow derivation from a single phonological representation. For example, in Mayak\il{Mayak} we saw that low vowel affixes may be consistently advanced, consistently retracted, or alternate between advanced and retracted forms -- behaviour that is trivially represented in \is{surface-to-surface}surface-oriented morph sets but requires some enrichment of the theory when attempting to postulate unique underlying representations for the three types of \is{morpheme!Mayak}morphemes with low vowels. A general pattern of tongue root harmony\is{harmony!Mayak}\is{tongue root!Mayak harmony} is apparent in the language, but whether it affects a particular form depends on whether any of the contributing morph sets contain both advanced and retracted vowels. The phonotactic\is{phonotactics} is independent of morph set structure;  \is{underlying representation}underlying representation models do not make this separation and so must create novel representations, whether rule-based or constraint-based.\largerpage


Finally, we return to learning, pointing to cases where reverse engineering\is{reverse engineering!acquisition} results in forms which would seem to constitute significant problems, languages exhibiting a wide variety of instances where there is no observed form from which all surface forms could be derived (e.g.\ Kinande,\il{Kinande} Mayak,\il{Mayak} Polish),\il{Polish} leading in turn to the postulation of forms of considerable abstractness.\is{abstractness}  In \il{Kinande}Kinande, for example, the adoption of rules of H tone spreading and delinking means that identifying a H tone on the surface requires  reverse engineering in order to ascertain which vowels must be considered underlyingly high toned (even though they are never high toned on the surface), while other vowels  are reverse engineered to have no underlying high tone (even though they are high toned on the surface).  In terms of learning, observations about surface tonal representations must lead to particular rules/constraints; based on these rules/constraints, \is{underlying representation}underlying representations must be constructed not because they ever occur for the \is{reverse engineering!morpheme}morpheme in question, but because the postulation of such a morpheme shape allows it to interact with the rule/constraint set and produce forms that are indeed attested. The requirements for learning are complex, and largely unaddressed; the same issues arise in modeling the perceptual\is{perception} processing\is{processing} of a phonetic string (\citealt{Boersma:2011}).\is{reverse engineering|)}


\subsection{Frequency in production and perception}\is{frequency|(}\is{production}\is{perception}

We have assumed that learners track the frequency\is{frequency} of occurrence of different items. Highly frequent polymorphic forms may even be represented as part of a stem's morph set: \{\ipa{lʊk, lʊkt}\down{\sc past}\}\down{\sc look}.\il{English} In such cases, when producing {\sc look-past}, the speaker would have both [\ipa{lʊkt}]\down{\sc look.past} and [\ipa{lʊk-t}]\down{\sc look-past}, giving two routes for identifying the appropriate form. This may lead to a difference in production\is{production!frequency} patterns -- or conversely in \is{perception!frequency} recognising lexical items -- based on the more frequent forms having multiple means for accessing the item.\is{frequency|)}



\section{Concluding remarks}
In setting out on this (ad)venture, our goal was to understand what innate\is{innateness|(} principles are absolutely necessary in order to represent adult phonological systems. Our explorations, dating from \citet{Pulleyblank:2006cls, Pulleyblank:2006wccfl, Mohanan+:2010}, have yet to reveal any compelling phonology-specific innate mechanisms. To the extent that our analyses integrate \is{morphology}morphology, there too we have yet to discover the need for  innate mechanisms specific to language. As we hope this monograph has shown, apparently complex data receive straightforward, transparent analysis within the Emergent framework. This chapter has sketched ways that Emergence might be extended to other types of phonological patterns, as well as to other domains of linguistic research. Because the general approach assumes a minimal  role for an innate linguistic faculty, the expectation is that there is no \is{grammar!components}``phonological component'', no ``morphological component'', etc.; the appearance of such silos must be epiphenomenal, a consequence of analysing each language on its own terms, rather than resulting from separate innate cognitive constructs.\is{cognition}\is{innateness|)} We are left wondering in what ways Emergence is relevant to the whole of language, syntax\is{syntax} and \is{semantics}semantics, perception\is{perception} and production,\is{production} usage and change, along with \is{morphology}morphology and phonology. 

