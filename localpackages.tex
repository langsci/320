\usepackage{tikz}%for flowcharts
\usetikzlibrary{shapes,arrows,fit,shapes.misc}

\usepackage{stmaryrd,soul}

\usepackage{tone} %for tone letters, etc.
\usepackage{framed,color}
\colorlet{shadecolor}{green!20}
\definecolor{light-gray}{gray}{0.75}


\usepackage{stackrel}% STACKING CENTERED TEXT
\newcommand{\XaboveY}[2]{\raisebox{-.9ex}{$\stackrel{\hbox{\sc #1}}{\hbox{\sc #2}}$}} % SYNTAX \XaboveY{x}{y}: TO STACK {\sc x} over {\sc y} in running text; x/y are centered with respect to each other in the gap and centered vertically in the text.

\newcommand{\xabovey}[2]{\raisebox{-.9ex}{$\stackrel{\hbox{#1}}{\hbox{#2}}$}}% SYNTAX \xabovey{x}{y}: TO STACK x over y (no small caps) in running text; x/y are centered with respect to each other in the gap and centered vertically in the text.

% STACKING 3 lines of CENTERED TEXT, with and without small caps. Note the bottommost comes first. Also, if you want brackets around them, for square brackets, \textbackslash bigg is sufficient, but for curly brackets, you need \textbackslash Bigg. For both, it is important to have a negative offset of -.9ex (\textbackslash raisebox\{-.9ex\}\{\textbackslash Bigg\}\}, etc. This could be built into the commands but the names would get messier.
\newcommand{\XaboveYaboveZ}[3]{\raisebox{-.9ex}{$\stackrel[{\hbox{\sc #1}}]{\hbox{\sc #2}}{\hbox{\sc #3}}$}}

\newcommand{\xaboveyabovez}[3]{\raisebox{-.9ex}{$\stackrel[\hbox{#1}]{\hbox{#2}}{\hbox{#3}}$}}

%To get 4 or more features in a matrix, embed xabovey; xaboveyabovez:
%we have defined {\it xabovey} and {\it xaboveyabovez}. So how do we get the 4 features in one matrix? By using {\it xabovey} with {\it xabovey-dorsal/voi} as ``x'' and {\it xabovey-obs/stop} as ``y''. And using {\it bmatrix} for beautiful square brackets! Tip: Keep careful track of your brackets if you do this. There are a lot of them!! Example follows:

%$\begin{bmatrix}\xabovey{\xabovey{dorsal}{voice}}{\xabovey{obstruent}{stop}}\end{bmatrix}$


\usepackage{enumerate}
\usepackage[normalem]{ulem}
\usepackage{multirow,bigdelim}

\usepackage{rotating}%for vertical text in tables; with \begin{sideways}revevant text\end{sideways}

\let\checkmark\undefined
\usepackage{dingbat}
\usepackage{mathrsfs} %for fancy R for "relation" $\mathscr{R}$
\usepackage{pifont}

\usepackage[shortlabels]{enumitem}

\usepackage{todonotes}
\usepackage{collectbox}

\usepackage{langsci-tbls}
\tcbuselibrary{theorems, breakable, skins, xparse}
\usepackage{langsci-optional}
\usepackage{langsci-gb4e}
\usepackage{multicol}
\usepackage{arydshln}
\usepackage[linguistics]{forest}
